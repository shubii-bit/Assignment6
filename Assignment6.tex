\documentclass{article}
\usepackage[utf8]{inputenc}
\usepackage{amsmath}
\usepackage{blkarray}
\title{Assignment 6}
\author{Shubham Shrivastava}
\date{\today}

\begin{document}

\maketitle

\section*{Question 1:}
Solve the system of equations:\\
\(\frac{2}{x}\)+\(\frac{3}{y}\)+\(\frac{10}{z}=4\)\\
\(\frac{4}{x}\)-\(\frac{6}{y}\)+\(\frac{5}{z}=1\)\\
\(\frac{6}{x}+\frac{9}{y}-\frac{20}{z}=2\)\\
\section*{Solution:}
let, \(\frac{1}{x}=p\),\(\frac{1}{y}=q\),\(\frac{1}{z}=r\)\\
then the given system as follows,\\

\begin{equation}
    2p+3q+10r= 4
\end{equation}\\
\begin{equation}
    4p-6q+5r=1
\end{equation}\\
\begin{equation}
    6p+9q-20r=2
\end{equation}\\

This can be written in the form of $AX=B$, where\\
$A$= $\begin{pmatrix} 2 & 3 & 10 \\4 & -6 & 5 \\6 & 9 & -20\\  \end{pmatrix}$, $X $ = \begin{pmatrix} p\\q \\r\\  \end{pmatrix}
and $B$ = \begin{pmatrix} 4\\1 \\2\\  \end{pmatrix}\\
$\\$
\begin{equation}
    \begin{split}
|A| & = 2(120-45)-3(-80-30)+10(36+360)\\
&=150+330+720\\
&=1200
\end{split}
\end{equation}

Now,\\
$\\$

Thus, A is non singular. Therefore, its inverse exists.\\
Now,\\
$A_1_1$=75,$A_1_2$=110,$A_1_3$=72\\
$A_2_1$=150,$A_2_2$=-100,$A_2_3$=0\\
$A_3_1$=75,$A_3_2$=32,$A_3_3$=-24\\
WKT;\\

   $A^{-1}$ = \(\frac{1}{|A|}\)adjA\\ 
=\(\frac{1}{1200}\)\begin{pmatrix} 75 & 150 & 75 \\110 & -100 & 30 \\72 & 0 & -24\\  \end{pmatrix}\\ 

$\\$

Now,\\
$X$=$A^(-1)B$\\
$=>$\begin{pmatrix} p\\q \\r\\  \end{pmatrix}=\(\frac{1}{1200}\)\begin{pmatrix} 75 & 150 & 75 \\110 & -100 & 30 \\72 & 0 & -24\\  \end{pmatrix}\begin{pmatrix} 4\\1\\2\\  \end{pmatrix}\\
=\(\frac{1}{1200}\)\begin{pmatrix} 300+150+150 \\440-100+60 \\288+0-48\\  \end{pmatrix}
=1200\begin{pmatrix} 600\\400 \\240\\  \end{pmatrix}= \begin{pmatrix} \frac{1}{2} \\\frac{1}{3} \\\frac{1}{5}\\  \end{pmatrix}\\
therefore\\
\begin{equation*}
    p=\frac{1}{2}, q=\frac{1}{3} and 
    r=\frac{1}{5}\\
    Hence, x=2,y=3 and z=5\\
\end{equation*}
\newpage
\section*{Question 2}
If a,b,c are in A.P, then the determinant\\
\begin{pmatrix} x+2 & x+3 & x+2a\\x+3 & x+4 & x+2b \\x+4 & x+5 & x+2c\\  \end{pmatrix} is \\
\section*{Solution}
Let, A = \begin{pmatrix} x+2 & x+3 & x+2a\\x+3 & x+4 & x+2b \\x+4 & x+5 & x+2c\\  \end{pmatrix}\\
\begin{equation*}
    R_2 => 2R_2 - R_1-R_3
\end{equation*}
=\(\frac{1}{2}\) \begin{pmatrix} x+2 & x+3 & x+2a\\0 & 0 & 2(2b-a-c) \\x+4 & x+5 & x+2c\\  \end{pmatrix}\\
Since a, b and c are in AP using 2b = a + c, we get\\
$A$= \begin{pmatrix} x+2 & x+3 & x+2a\\0 & 0 & 0 \\x+4 & x+5 & x+2c\\  \end{pmatrix}\\
Since, all elements of $R_2$ are zero\\



\newpage
\section*{Question 3}
Two farmers Ramkishan and Gurcharan Singh
cultivates only three varieties of rice namely
Basmati, Permal and Naura. The sale (in
Rupees) of these varieties of rice by both
the farmers in the month of September and
October are given by the following matrices
A and B.
September Sales(in Rupees)\\
\[ A= 
\begin{blockarray}{cccc}
Basmati & Permal & Naura \\
\begin{block}{(ccc)c}
  10, 000 & 20, 000& 30, 000 & Ramakishan\\
50, 000 &30, 000 &10, 000 & Gurucharan Singh\\
\end{block}
\end{blockarray}
 \]
October sales (in Rupees)\\
\[ B = 
\begin{blockarray}{cccc}
Basmati & Permal & Naura \\
\begin{block}{(ccc)c}
  5,000 & 10,000& 6,000 & Ramakishan\\
50,000 &10,000 &10,000 & Gurucharan Singh\\
\end{block}
\end{blockarray}
 \]
(i) Find the combined sales in September and
October for each farmer in each variety.\\
(ii) Find the decrease in sales from September
to October.\\
(iii) If both farmers receive 2% profit on gross
sales, compute the profit for each farmer and
for each variety sold in October.
\section*{Solution}
(i) The combined sales from Setember to October will be \\
\begin{equation*}
    \begin{split}
        A+B &= \begin{pmatrix}
        10,000 & 20,000& 30, 000\\
        50, 000 &30, 000 &10, 000 
        \end{pmatrix} + \begin{pmatrix}
        5,000 & 10,000& 6,000\\
50,000 &10,000 &10,000\\
        \end{pmatrix}\\
        &= \begin{pmatrix}
         10, 000+5,000 & 20, 000+10,000& 30, 000+6,000\\
        50, 000+20,000 &30, 000+10,000 &10, 000+10,000 
        \end{pmatrix}\\
        &= \begin{pmatrix}
        15,000 & 30,000 & 36,000\\
        70,000 & 40,000 & 20,000
        \end{pmatrix}
    \end{split}
\end{equation*}
thefore the combined sales would be as:
\[ A+B = 
\begin{blockarray}{cccc}
Basmati & Permal & Naura \\
\begin{block}{(ccc)c}
  15,000 & 30,000 & 36,000 & Ramakishan\\
70,000 & 40,000 & 20,000 & Gurucharan Singh\\
\end{block}
\end{blockarray}
 \]


\newpage
(ii) Decrease in sales from september to october will be $A-B$\\
\begin{equation*}
\begin{split}
    A-B &= \begin{pmatrix} 
            10000 & 20000 & 30000\\50000 & 30000 &10000 \\ \end{pmatrix}-\begin{pmatrix} 5000 & 10000 & 6000\\20000 & 10000 & 10000 \\ 
            \end{pmatrix}\\
        &=\begin{pmatrix} 
             10000-5000 & 20000-10000 & 30000-6000\\50000-20000 & 30000-10000 &10000-10000 \\ 
        \end{pmatrix}\\
        &=\begin{pmatrix} 
            5000 & 10000 & 24000\\30000 & 20000 & 0 \\ \end{pmatrix}\\
\end{split}
    
\end{equation*}


Hence decrease in sales from september to october\\
\[\begin{blockarray}{cccc}
Basmati & Permal & Naura \\
\begin{block}{(ccc)c}
  5,000 & 10,000& 24,000 & Ramakishan\\
30,000 &20,000 & 0 & Gurucharan Singh\\
\end{block}
\end{blockarray}
\]
(iii)

\begin{equation*}
    \begin{split}
         Profit &= 2 \% *(sales in october)\\
              &=\frac{2}{100} * B\\
              &=0.02 * B\\
              & =0.02 \begin{pmatrix} 
                        5000 & 10000 & 6000\\20000 & 10000 & 10000 \\ 
                    \end{pmatrix}\\
             &=\begin{pmatrix} 
                    0.02*5000 & 0.02*10000 & 0.02*6000\\0.02*20000 & 0.02*10000 & 0.02*10000 \\ 
                \end{pmatrix}\\
           & =\begin{pmatrix} 
                    100 & 200 & 120\\400 & 200 &\\
                \end{pmatrix}
    \end{split}
\end{equation*}
       
            
    

Hence,
\[ profit = 
\begin{blockarray}{cccc}
Basmati & Permal & Naura \\
\begin{block}{(ccc)c}
  100 & 200 & 120 & Ramakishan\\
400 & 200 & 200 & Gurucharan Singh\\
\end{block}
\end{blockarray}
 \]
Hence,\\
Ramkrishan recieves\\

RS 100 profit on sale of Basmati Rice,\\

RS 200 profit on sale of Permal\\

&RS 120 profit in the sale of Naura in the month of October\\ \\
Hence,\\
Gurucharan Singh recieves \\

Rs.400 profit on sales of basmati rice\\

Rs 200 profit on sales of permal\\

&Rs.200 profit on sales of nora in the month of october\\


\end{document}
